\documentclass[12pt]{article}
\usepackage[warningsoff,address,pgf]{stargate}
\usepackage[textwidth=18cm,textheight=24cm]{geometry}
\usepackage{ctex}
\usepackage{tabularray}
\usepackage[skins,breakable,listings]{tcolorbox}
\usepackage{hyperref}
\hypersetup{colorlinks}
\def\tableautorefname{表}

\definecolor{examplebg}{RGB}{252, 252, 252}
\definecolor{examplelblue}{RGB}{233, 243, 255}
\definecolor{exampleblue}{RGB}{0, 123, 255}
\definecolor{examplelred}{RGB}{255, 242, 242}
\definecolor{examplered}{RGB}{220, 53, 69}
\newcounter{example}
\tcbset{
  breakable,
  example base/.style={
    enhanced jigsaw,
    drop fuzzy shadow southeast,
    sharp corners,
    boxrule=0pt,
    fonttitle={\bfseries},
    coltitle=black,
    left=2mm,right=2mm,top=1mm,bottom=1mm,
    fontupper=\normalfont,
    colback=examplebg,
  },
}
\newtcblisting[use counter=example]{example}[1][]{%
  example base,
  colbacktitle=examplelblue,
  before title=例 \thetcbcounter:, title=~,
  borderline west={1.5pt}{0pt}{exampleblue},
  #1}
\newtcolorbox{setting}[2][]{%
  example base,
  colbacktitle=examplelred,
  borderline west={1.5pt}{0pt}{examplered},
  title={#2},
  #1}

\providecommand{\pkg}{\textsf}
\providecommand{\tikzlogo}{Ti\emph{k}Z}
\providecommand{\cs}[1]{\texttt{\textbackslash#1}}
\providecommand{\meta}[1]{$\langle$\textit{#1}$\rangle$}
\providecommand{\opt}{\texttt}
\providecommand{\marg}[1]{\texttt\{\meta{#1}\texttt\}}
\providecommand{\oarg}[1]{\texttt[\meta{#1}\texttt]}

\begin{document}

\vspace*{1em}
\begin{center}
\bfseries\Huge\StargateLogo[.6]
\end{center}
\vspace{1em}

\begin{tikzpicture}
\StargateDHD[sgdhd origin=30,sgdhd lock=\StargateAddressOfName{DESTINY}]
\StargateGate[sggate scale=2.3]
\end{tikzpicture}

\bigskip

{\bfseries\large\pkg{星际之门}(\pkg{stargate})元素的一种实现
\\ \hbox{}\hfill ——使用 \tikzlogo 和 Stargate SG1 Address Glyph 字体}

\hfill (雾月,Longaster \quad \today)

\clearpage

\begin{example}[listing side text,righthand ratio=.44,title=绘制DHD (Dial Home Device)]
Dial The Gate! 

FROM (从)(CHULAK): \StargateAddressByName{CHULAK}

Point Of Origin (起点): \StargateGlyph{7}

\tikz\StargateDHD[
  sgdhd origin=7,
  sgdhd lock={\StargateAddressOfName{EARTH}},
];

TO (到)(EARTH): \StargateAddressByName{EARTH}
\end{example}

\begin{example}[title=绘制星际之门]
绘制星际之门:\tikz[baseline]\StargateGate;
\end{example}

在加载 \pkg{stargate} 宏包时使用 \opt{pgf} 选项将加载 \pkg{tikz} 宏包,并定义这两个命令。

Logo: \cs{StargateLogo}\oarg{width},这个可选参数控制外侧线宽。

\verb|\StargateLogo[0.25]|\StargateLogo[0.25]。

\begin{example}[title=星门字形,listing side text,lefthand ratio=.6]
可通过 \verb|\StargateGlyph| 选择字形:
\Huge \StargateGlyph{1}\StargateGlyph{2}
\StargateGlyph{39}\StargateGlyph{40} \\
\StargateGlyph{Earth}\StargateGlyph{Crater}
\StargateGlyph{Leo}\StargateGlyph{Abydos} 
\end{example}

可用的字形参考\autoref{tab:glyph},可以使用名称或数字。

\makeatletter
\stargate@tmpctn=0
\def\TMP{}
\def\tmp@addto#1#2{
  \advance\stargate@tmpctn by 1
  \g@addto@macro\TMP{%
    {\begin{tabular}{cc}
      \Huge\StargateGlyph{#1}\\
      \small #1/#2\end{tabular}}}
  \ifnum\stargate@tmpctn=5
    \g@addto@macro\TMP{ \\ }
    \stargate@tmpctn=0
  \else\g@addto@macro\TMP{ & }
  \fi}
\stargate@zip@\stargate@glyph@list\stargate@glyph@num\tmp@addto
\makeatother
\begin{table}[!ht]
\caption{可用的字形(Stargate Glyph)}\label{tab:glyph}\vspace{8pt}
\begin{tblr}[expand=\TMP]{
  colspec={*{5}{|X[c]}|},hlines,
  rows={abovesep+=5pt},columns={colsep=-5pt},
}
\TMP 
\end{tblr}

\begin{tblr}{colspec={|c|c|X[c]|X[c]|X[c]|c|},hlines}
原义 & Scorpio/9 & Cra/10 & Mic/14 & Capricorn/15 & PiscesAustrinus/16 \\
同义 & Scorpius & CoronaAustralis & Microscopium & Capricornus & PiscisAustrinus
\end{tblr}
\end{table}

你可以通过 \cs{StargateGlyphName}\marg{integer} 来获得字形的名称,它是可扩展的。

“同义”指这两个词的作用完全相同,你可以使用 \cs{stargatesetalian}\marg{alian}\marg{origin} 来设置自己的同义词。

也支持使用 \cs{StargaetGlyph}\meta{name} 来引用:
\begin{example}[]
\Huge \StargateGlyphEarth \StargateGlyphCrater \StargateGlyphLeo \StargateGlyphAbydos
\end{example}

可以使用 \verb|\stargate| 和 \verb|\stargatefamily|,它们类似 \cs{texttt} 和 \cs{ttfamily}。可用的符号为 \verb|A-Z| 和 \verb|a-n|,依次代表了40个可用的字形。
\begin{example}[]
\stargate{ABCDEFGHIJKLMNOPQRSTUVWXYZabcdefghijklmn}
\end{example}

定义了18个起点(16个有效),可以通过 \cs{StargateOrigin} 引用,全部的18个起点见\autoref{tab:origin}。

你可以通过 \cs{StargatePointName}\marg{integer} 来获得起点的名称,它是可扩展的。

\begin{example}[title=起点,listing side text,lefthand ratio=.6]
\Huge \StargateOrigin{1}\StargateOrigin{2}
\StargateOrigin{17}\StargateOrigin{18} \\
\StargateOrigin{Earth}\StargateOrigin{Abydos}
\StargateOrigin{Tagrea}\StargateOrigin{P4M-328}
\end{example}

\makeatletter
\def\TMP{}
\stargate@tmpctn=0
\def\tmp@addto#1#2{
  \advance\stargate@tmpctn by 1
  \g@addto@macro\TMP{%
    {\begin{tabular}{cc}
      \Huge\StargateOrigin{#1}\\
      \small \detokenize{#1}/#2\end{tabular}}}
  \ifnum\stargate@tmpctn=5
    \g@addto@macro\TMP{ \\ }
    \stargate@tmpctn=0
  \else\g@addto@macro\TMP{ & }
  \fi}
\stargate@zip@\stargate@origin@point\stargate@origin@num\tmp@addto
\makeatother
\begin{table}[!ht]
\caption{可用的起点(Point of Origin)}\label{tab:origin}\vspace{8pt}
\begin{tblr}[expand=\TMP]{
  colspec={*{5}{|X[c]}|},hlines,
  rows={abovesep+=5pt},columns={colsep=-5pt},
}
\TMP 
\end{tblr}
\end{table}

可以使用列表或地址名来输出地址:
\begin{example}[listing side text,lefthand ratio=.6,title=输出地址]
\Huge \StargateAddress{9,18,27,15,21,36} \\
\StargateAddressByName{FINAL DESTINATION}
\end{example}

也可以使用 \cs{StargateAddAddress}\marg{name}\marg{address list} 来定义新的地址,新的名称将覆盖旧的。

你可以使用 \cs{StargateAddressOfName}\marg{address name} 获得对应的地址,它是可扩展的。

在加载 \pkg{stargate} 宏包时使用 \opt{address} 选项,将预定义一些已有的地址,这些地址见\autoref{tab:address}。

\ExplSyntaxOn
\def\TMP{}
\char_set_catcode_alignment:N \&
\prop_map_inline:Nn \g__stargate_address_prop
  {\tl_put_right:Nn \TMP 
    {\texttt{#1} & #2 & \LARGE\StargateAddressByName{#1} \\ }}
\ExplSyntaxOff
\DefTblrTemplate{contfoot-text}{default}{\tiny (接下页)}
\DefTblrTemplate{conthead-text}{default}{\tiny (接上页)}
\begin{longtblr}[
  caption=预定义的地址(Address),label=tab:address,expand=\TMP,
]{colspec={|c|c|c|},hlines,hline{1,2,Z}={1pt},vlines={1pt},rowhead=1}
名称 & 地址列表 & 地址 \\
\TMP
\end{longtblr}

\opt{pgf} 宏包选项定义了 \cs{StargateGate} 和 \cs{StargateDHD} 命令,用于绘制星际之门和DHD,必须用在 \tikzlogo 环境中。
\cs{StargateGate} 和 \cs{StargateDHD} 可用的设置:
\begin{setting}{可用设置}
{\obeylines
星门:
\verb|/tikz/sggate center point=|\marg{coordinate}
\verb|/tikz/sggate scale=|\marg{integer}
\verb|/tikz/sggate node glyph/.append style=|\marg{tikz options}
\verb|(sg Gate Center)|、\verb|(sg Gate Glyph-1)|、\verb|...|、\verb|(sg Gate Glyph-39)| 坐标点
DHD:
\verb|/tikz/sgdhd center point=|\marg{coordinate}
\verb|/tikz/sgdhd scale=|\marg{integer}
\verb|/tikz/sgdhd glyph/.append style=|\marg{tikz options}
\verb|/tikz/sgdhd origin=|\marg{integer},设置起点,范围为1-40
\verb|/tikz/sgdhd origin by name=|\marg{name},通过名称设置起点,不能是同义
\verb|/tikz/sgdhd button base/.append style=|\marg{tikz options},设置中心按钮底座样式
\verb|/tikz/sgdhd button/.append style=|\marg{tikz options},设置中心按钮样式
\verb|/tikz/sgdhd button color=|\marg{color expr}
\verb|/tikz/sgdhd button glyph/.append style=|\marg{tikz options}
\verb|/tikz/sgdhd base/.append style=|\marg{tikz options},设置地盘样式
\verb|/tikz/sgdhd base base/.append style=|\marg{tikz options},设置底盘背景样式
\verb|/tikz/sgdhd locked color=|\marg{color expr},设置锁定的按钮的颜色
\verb|/tikz/sgdhd lock=|\marg{address list},设置要锁定的按钮1-39,小的在外侧,大的在内侧,以通过 \cs{StargateAddressOfName} 来通过名字引用。由于 Abydos(40) 在《星际之门》影视作品中从未作为可输入的地址,因此,这里不支持使用它
\verb|/tikz/sgdhd lock*=|\marg{address list}\marg{tikz options}
\verb|/tikz/sgdhd inner button 1/.append style=|\marg{tikz options}
\verb|...|
\verb|/tikz/sgdhd inner button 18/.append style=|\marg{tikz options}
\verb|/tikz/sgdhd outer button 1/.append style=|\marg{tikz options}
\verb|...|
\verb|/tikz/sgdhd outer button 18/.append style=|\marg{tikz options}
\verb|(sg DHD Center)| 坐标点
}
\end{setting}

《星际之门》(\textit{STARGATE})由米高梅和Scifi出品,影视作品系列包含:
\begin{itemize}
  \item 1994年的同名电影;
  \item 1997年-2007年的电视剧 Stargate: SG-1,10季,共216集,每集大约44分钟;
  \item 2004年-2009年的电视剧 Stargate: Atlantis(SGA,星际之门:亚特兰蒂斯),5季,共100集(未完),每集大约44分钟;
  \item 2009年-2011年的电视剧 Stargate: Universe(SGU,星际之门:宇宙),2季,共40集(未完),每集大约43分钟;
  \item 2008年的两部电影 Stargate: The Ark of Truth(星际之门:真理之箱),97分钟,Stargate: Continuum(星际之门:时空连续),93分钟。
  \item 以及2018年的网剧 Stargate Origins(星际之门:起源),共10集。
\end{itemize}

电影和电视剧皆为续作,SG-1和SGA有交叉。总的来说不输于《星际迷航》系列。对于科幻迷来说是一部不得不看的好剧(前提是能看完17季 + 3部电影)。

已经近10年没有正剧了,听说今年有活动,期待《星际之门》归来!

《星际之门》的更多信息请看:\url{https://stargate.fandom.com/wiki/Stargate_Wiki}。

影视作品中出现的星门地址请参考:
\url{https://stargate.fandom.com/wiki/Glyph}, \\
\url{https://www.rdanderson.com/stargate/glyphs/index.htm}, \\
\url{https://thestargateproject.com/address-book/},
\pkg{stargate} 宏包已全部包含。

\end{document}